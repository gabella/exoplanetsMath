
% Default to the notebook output style

    


% Inherit from the specified cell style.




    
\documentclass[11pt]{article}

    
    
    \usepackage[T1]{fontenc}
    % Nicer default font (+ math font) than Computer Modern for most use cases
    \usepackage{mathpazo}

    % Basic figure setup, for now with no caption control since it's done
    % automatically by Pandoc (which extracts ![](path) syntax from Markdown).
    \usepackage{graphicx}
    % We will generate all images so they have a width \maxwidth. This means
    % that they will get their normal width if they fit onto the page, but
    % are scaled down if they would overflow the margins.
    \makeatletter
    \def\maxwidth{\ifdim\Gin@nat@width>\linewidth\linewidth
    \else\Gin@nat@width\fi}
    \makeatother
    \let\Oldincludegraphics\includegraphics
    % Set max figure width to be 80% of text width, for now hardcoded.
    \renewcommand{\includegraphics}[1]{\Oldincludegraphics[width=.8\maxwidth]{#1}}
    % Ensure that by default, figures have no caption (until we provide a
    % proper Figure object with a Caption API and a way to capture that
    % in the conversion process - todo).
    \usepackage{caption}
    \DeclareCaptionLabelFormat{nolabel}{}
    \captionsetup{labelformat=nolabel}

    \usepackage{adjustbox} % Used to constrain images to a maximum size 
    \usepackage{xcolor} % Allow colors to be defined
    \usepackage{enumerate} % Needed for markdown enumerations to work
    \usepackage{geometry} % Used to adjust the document margins
    \usepackage{amsmath} % Equations
    \usepackage{amssymb} % Equations
    \usepackage{textcomp} % defines textquotesingle
    % Hack from http://tex.stackexchange.com/a/47451/13684:
    \AtBeginDocument{%
        \def\PYZsq{\textquotesingle}% Upright quotes in Pygmentized code
    }
    \usepackage{upquote} % Upright quotes for verbatim code
    \usepackage{eurosym} % defines \euro
    \usepackage[mathletters]{ucs} % Extended unicode (utf-8) support
    \usepackage[utf8x]{inputenc} % Allow utf-8 characters in the tex document
    \usepackage{fancyvrb} % verbatim replacement that allows latex
    \usepackage{grffile} % extends the file name processing of package graphics 
                         % to support a larger range 
    % The hyperref package gives us a pdf with properly built
    % internal navigation ('pdf bookmarks' for the table of contents,
    % internal cross-reference links, web links for URLs, etc.)
    \usepackage{hyperref}
    \usepackage{longtable} % longtable support required by pandoc >1.10
    \usepackage{booktabs}  % table support for pandoc > 1.12.2
    \usepackage[inline]{enumitem} % IRkernel/repr support (it uses the enumerate* environment)
    \usepackage[normalem]{ulem} % ulem is needed to support strikethroughs (\sout)
                                % normalem makes italics be italics, not underlines
    

    
    
    % Colors for the hyperref package
    \definecolor{urlcolor}{rgb}{0,.145,.698}
    \definecolor{linkcolor}{rgb}{.71,0.21,0.01}
    \definecolor{citecolor}{rgb}{.12,.54,.11}

    % ANSI colors
    \definecolor{ansi-black}{HTML}{3E424D}
    \definecolor{ansi-black-intense}{HTML}{282C36}
    \definecolor{ansi-red}{HTML}{E75C58}
    \definecolor{ansi-red-intense}{HTML}{B22B31}
    \definecolor{ansi-green}{HTML}{00A250}
    \definecolor{ansi-green-intense}{HTML}{007427}
    \definecolor{ansi-yellow}{HTML}{DDB62B}
    \definecolor{ansi-yellow-intense}{HTML}{B27D12}
    \definecolor{ansi-blue}{HTML}{208FFB}
    \definecolor{ansi-blue-intense}{HTML}{0065CA}
    \definecolor{ansi-magenta}{HTML}{D160C4}
    \definecolor{ansi-magenta-intense}{HTML}{A03196}
    \definecolor{ansi-cyan}{HTML}{60C6C8}
    \definecolor{ansi-cyan-intense}{HTML}{258F8F}
    \definecolor{ansi-white}{HTML}{C5C1B4}
    \definecolor{ansi-white-intense}{HTML}{A1A6B2}

    % commands and environments needed by pandoc snippets
    % extracted from the output of `pandoc -s`
    \providecommand{\tightlist}{%
      \setlength{\itemsep}{0pt}\setlength{\parskip}{0pt}}
    \DefineVerbatimEnvironment{Highlighting}{Verbatim}{commandchars=\\\{\}}
    % Add ',fontsize=\small' for more characters per line
    \newenvironment{Shaded}{}{}
    \newcommand{\KeywordTok}[1]{\textcolor[rgb]{0.00,0.44,0.13}{\textbf{{#1}}}}
    \newcommand{\DataTypeTok}[1]{\textcolor[rgb]{0.56,0.13,0.00}{{#1}}}
    \newcommand{\DecValTok}[1]{\textcolor[rgb]{0.25,0.63,0.44}{{#1}}}
    \newcommand{\BaseNTok}[1]{\textcolor[rgb]{0.25,0.63,0.44}{{#1}}}
    \newcommand{\FloatTok}[1]{\textcolor[rgb]{0.25,0.63,0.44}{{#1}}}
    \newcommand{\CharTok}[1]{\textcolor[rgb]{0.25,0.44,0.63}{{#1}}}
    \newcommand{\StringTok}[1]{\textcolor[rgb]{0.25,0.44,0.63}{{#1}}}
    \newcommand{\CommentTok}[1]{\textcolor[rgb]{0.38,0.63,0.69}{\textit{{#1}}}}
    \newcommand{\OtherTok}[1]{\textcolor[rgb]{0.00,0.44,0.13}{{#1}}}
    \newcommand{\AlertTok}[1]{\textcolor[rgb]{1.00,0.00,0.00}{\textbf{{#1}}}}
    \newcommand{\FunctionTok}[1]{\textcolor[rgb]{0.02,0.16,0.49}{{#1}}}
    \newcommand{\RegionMarkerTok}[1]{{#1}}
    \newcommand{\ErrorTok}[1]{\textcolor[rgb]{1.00,0.00,0.00}{\textbf{{#1}}}}
    \newcommand{\NormalTok}[1]{{#1}}
    
    % Additional commands for more recent versions of Pandoc
    \newcommand{\ConstantTok}[1]{\textcolor[rgb]{0.53,0.00,0.00}{{#1}}}
    \newcommand{\SpecialCharTok}[1]{\textcolor[rgb]{0.25,0.44,0.63}{{#1}}}
    \newcommand{\VerbatimStringTok}[1]{\textcolor[rgb]{0.25,0.44,0.63}{{#1}}}
    \newcommand{\SpecialStringTok}[1]{\textcolor[rgb]{0.73,0.40,0.53}{{#1}}}
    \newcommand{\ImportTok}[1]{{#1}}
    \newcommand{\DocumentationTok}[1]{\textcolor[rgb]{0.73,0.13,0.13}{\textit{{#1}}}}
    \newcommand{\AnnotationTok}[1]{\textcolor[rgb]{0.38,0.63,0.69}{\textbf{\textit{{#1}}}}}
    \newcommand{\CommentVarTok}[1]{\textcolor[rgb]{0.38,0.63,0.69}{\textbf{\textit{{#1}}}}}
    \newcommand{\VariableTok}[1]{\textcolor[rgb]{0.10,0.09,0.49}{{#1}}}
    \newcommand{\ControlFlowTok}[1]{\textcolor[rgb]{0.00,0.44,0.13}{\textbf{{#1}}}}
    \newcommand{\OperatorTok}[1]{\textcolor[rgb]{0.40,0.40,0.40}{{#1}}}
    \newcommand{\BuiltInTok}[1]{{#1}}
    \newcommand{\ExtensionTok}[1]{{#1}}
    \newcommand{\PreprocessorTok}[1]{\textcolor[rgb]{0.74,0.48,0.00}{{#1}}}
    \newcommand{\AttributeTok}[1]{\textcolor[rgb]{0.49,0.56,0.16}{{#1}}}
    \newcommand{\InformationTok}[1]{\textcolor[rgb]{0.38,0.63,0.69}{\textbf{\textit{{#1}}}}}
    \newcommand{\WarningTok}[1]{\textcolor[rgb]{0.38,0.63,0.69}{\textbf{\textit{{#1}}}}}
    
    
    % Define a nice break command that doesn't care if a line doesn't already
    % exist.
    \def\br{\hspace*{\fill} \\* }
    % Math Jax compatability definitions
    \def\gt{>}
    \def\lt{<}
    % Document parameters
    \title{ExopDBase}
    
    
    

    % Pygments definitions
    
\makeatletter
\def\PY@reset{\let\PY@it=\relax \let\PY@bf=\relax%
    \let\PY@ul=\relax \let\PY@tc=\relax%
    \let\PY@bc=\relax \let\PY@ff=\relax}
\def\PY@tok#1{\csname PY@tok@#1\endcsname}
\def\PY@toks#1+{\ifx\relax#1\empty\else%
    \PY@tok{#1}\expandafter\PY@toks\fi}
\def\PY@do#1{\PY@bc{\PY@tc{\PY@ul{%
    \PY@it{\PY@bf{\PY@ff{#1}}}}}}}
\def\PY#1#2{\PY@reset\PY@toks#1+\relax+\PY@do{#2}}

\expandafter\def\csname PY@tok@w\endcsname{\def\PY@tc##1{\textcolor[rgb]{0.73,0.73,0.73}{##1}}}
\expandafter\def\csname PY@tok@c\endcsname{\let\PY@it=\textit\def\PY@tc##1{\textcolor[rgb]{0.25,0.50,0.50}{##1}}}
\expandafter\def\csname PY@tok@cp\endcsname{\def\PY@tc##1{\textcolor[rgb]{0.74,0.48,0.00}{##1}}}
\expandafter\def\csname PY@tok@k\endcsname{\let\PY@bf=\textbf\def\PY@tc##1{\textcolor[rgb]{0.00,0.50,0.00}{##1}}}
\expandafter\def\csname PY@tok@kp\endcsname{\def\PY@tc##1{\textcolor[rgb]{0.00,0.50,0.00}{##1}}}
\expandafter\def\csname PY@tok@kt\endcsname{\def\PY@tc##1{\textcolor[rgb]{0.69,0.00,0.25}{##1}}}
\expandafter\def\csname PY@tok@o\endcsname{\def\PY@tc##1{\textcolor[rgb]{0.40,0.40,0.40}{##1}}}
\expandafter\def\csname PY@tok@ow\endcsname{\let\PY@bf=\textbf\def\PY@tc##1{\textcolor[rgb]{0.67,0.13,1.00}{##1}}}
\expandafter\def\csname PY@tok@nb\endcsname{\def\PY@tc##1{\textcolor[rgb]{0.00,0.50,0.00}{##1}}}
\expandafter\def\csname PY@tok@nf\endcsname{\def\PY@tc##1{\textcolor[rgb]{0.00,0.00,1.00}{##1}}}
\expandafter\def\csname PY@tok@nc\endcsname{\let\PY@bf=\textbf\def\PY@tc##1{\textcolor[rgb]{0.00,0.00,1.00}{##1}}}
\expandafter\def\csname PY@tok@nn\endcsname{\let\PY@bf=\textbf\def\PY@tc##1{\textcolor[rgb]{0.00,0.00,1.00}{##1}}}
\expandafter\def\csname PY@tok@ne\endcsname{\let\PY@bf=\textbf\def\PY@tc##1{\textcolor[rgb]{0.82,0.25,0.23}{##1}}}
\expandafter\def\csname PY@tok@nv\endcsname{\def\PY@tc##1{\textcolor[rgb]{0.10,0.09,0.49}{##1}}}
\expandafter\def\csname PY@tok@no\endcsname{\def\PY@tc##1{\textcolor[rgb]{0.53,0.00,0.00}{##1}}}
\expandafter\def\csname PY@tok@nl\endcsname{\def\PY@tc##1{\textcolor[rgb]{0.63,0.63,0.00}{##1}}}
\expandafter\def\csname PY@tok@ni\endcsname{\let\PY@bf=\textbf\def\PY@tc##1{\textcolor[rgb]{0.60,0.60,0.60}{##1}}}
\expandafter\def\csname PY@tok@na\endcsname{\def\PY@tc##1{\textcolor[rgb]{0.49,0.56,0.16}{##1}}}
\expandafter\def\csname PY@tok@nt\endcsname{\let\PY@bf=\textbf\def\PY@tc##1{\textcolor[rgb]{0.00,0.50,0.00}{##1}}}
\expandafter\def\csname PY@tok@nd\endcsname{\def\PY@tc##1{\textcolor[rgb]{0.67,0.13,1.00}{##1}}}
\expandafter\def\csname PY@tok@s\endcsname{\def\PY@tc##1{\textcolor[rgb]{0.73,0.13,0.13}{##1}}}
\expandafter\def\csname PY@tok@sd\endcsname{\let\PY@it=\textit\def\PY@tc##1{\textcolor[rgb]{0.73,0.13,0.13}{##1}}}
\expandafter\def\csname PY@tok@si\endcsname{\let\PY@bf=\textbf\def\PY@tc##1{\textcolor[rgb]{0.73,0.40,0.53}{##1}}}
\expandafter\def\csname PY@tok@se\endcsname{\let\PY@bf=\textbf\def\PY@tc##1{\textcolor[rgb]{0.73,0.40,0.13}{##1}}}
\expandafter\def\csname PY@tok@sr\endcsname{\def\PY@tc##1{\textcolor[rgb]{0.73,0.40,0.53}{##1}}}
\expandafter\def\csname PY@tok@ss\endcsname{\def\PY@tc##1{\textcolor[rgb]{0.10,0.09,0.49}{##1}}}
\expandafter\def\csname PY@tok@sx\endcsname{\def\PY@tc##1{\textcolor[rgb]{0.00,0.50,0.00}{##1}}}
\expandafter\def\csname PY@tok@m\endcsname{\def\PY@tc##1{\textcolor[rgb]{0.40,0.40,0.40}{##1}}}
\expandafter\def\csname PY@tok@gh\endcsname{\let\PY@bf=\textbf\def\PY@tc##1{\textcolor[rgb]{0.00,0.00,0.50}{##1}}}
\expandafter\def\csname PY@tok@gu\endcsname{\let\PY@bf=\textbf\def\PY@tc##1{\textcolor[rgb]{0.50,0.00,0.50}{##1}}}
\expandafter\def\csname PY@tok@gd\endcsname{\def\PY@tc##1{\textcolor[rgb]{0.63,0.00,0.00}{##1}}}
\expandafter\def\csname PY@tok@gi\endcsname{\def\PY@tc##1{\textcolor[rgb]{0.00,0.63,0.00}{##1}}}
\expandafter\def\csname PY@tok@gr\endcsname{\def\PY@tc##1{\textcolor[rgb]{1.00,0.00,0.00}{##1}}}
\expandafter\def\csname PY@tok@ge\endcsname{\let\PY@it=\textit}
\expandafter\def\csname PY@tok@gs\endcsname{\let\PY@bf=\textbf}
\expandafter\def\csname PY@tok@gp\endcsname{\let\PY@bf=\textbf\def\PY@tc##1{\textcolor[rgb]{0.00,0.00,0.50}{##1}}}
\expandafter\def\csname PY@tok@go\endcsname{\def\PY@tc##1{\textcolor[rgb]{0.53,0.53,0.53}{##1}}}
\expandafter\def\csname PY@tok@gt\endcsname{\def\PY@tc##1{\textcolor[rgb]{0.00,0.27,0.87}{##1}}}
\expandafter\def\csname PY@tok@err\endcsname{\def\PY@bc##1{\setlength{\fboxsep}{0pt}\fcolorbox[rgb]{1.00,0.00,0.00}{1,1,1}{\strut ##1}}}
\expandafter\def\csname PY@tok@kc\endcsname{\let\PY@bf=\textbf\def\PY@tc##1{\textcolor[rgb]{0.00,0.50,0.00}{##1}}}
\expandafter\def\csname PY@tok@kd\endcsname{\let\PY@bf=\textbf\def\PY@tc##1{\textcolor[rgb]{0.00,0.50,0.00}{##1}}}
\expandafter\def\csname PY@tok@kn\endcsname{\let\PY@bf=\textbf\def\PY@tc##1{\textcolor[rgb]{0.00,0.50,0.00}{##1}}}
\expandafter\def\csname PY@tok@kr\endcsname{\let\PY@bf=\textbf\def\PY@tc##1{\textcolor[rgb]{0.00,0.50,0.00}{##1}}}
\expandafter\def\csname PY@tok@bp\endcsname{\def\PY@tc##1{\textcolor[rgb]{0.00,0.50,0.00}{##1}}}
\expandafter\def\csname PY@tok@fm\endcsname{\def\PY@tc##1{\textcolor[rgb]{0.00,0.00,1.00}{##1}}}
\expandafter\def\csname PY@tok@vc\endcsname{\def\PY@tc##1{\textcolor[rgb]{0.10,0.09,0.49}{##1}}}
\expandafter\def\csname PY@tok@vg\endcsname{\def\PY@tc##1{\textcolor[rgb]{0.10,0.09,0.49}{##1}}}
\expandafter\def\csname PY@tok@vi\endcsname{\def\PY@tc##1{\textcolor[rgb]{0.10,0.09,0.49}{##1}}}
\expandafter\def\csname PY@tok@vm\endcsname{\def\PY@tc##1{\textcolor[rgb]{0.10,0.09,0.49}{##1}}}
\expandafter\def\csname PY@tok@sa\endcsname{\def\PY@tc##1{\textcolor[rgb]{0.73,0.13,0.13}{##1}}}
\expandafter\def\csname PY@tok@sb\endcsname{\def\PY@tc##1{\textcolor[rgb]{0.73,0.13,0.13}{##1}}}
\expandafter\def\csname PY@tok@sc\endcsname{\def\PY@tc##1{\textcolor[rgb]{0.73,0.13,0.13}{##1}}}
\expandafter\def\csname PY@tok@dl\endcsname{\def\PY@tc##1{\textcolor[rgb]{0.73,0.13,0.13}{##1}}}
\expandafter\def\csname PY@tok@s2\endcsname{\def\PY@tc##1{\textcolor[rgb]{0.73,0.13,0.13}{##1}}}
\expandafter\def\csname PY@tok@sh\endcsname{\def\PY@tc##1{\textcolor[rgb]{0.73,0.13,0.13}{##1}}}
\expandafter\def\csname PY@tok@s1\endcsname{\def\PY@tc##1{\textcolor[rgb]{0.73,0.13,0.13}{##1}}}
\expandafter\def\csname PY@tok@mb\endcsname{\def\PY@tc##1{\textcolor[rgb]{0.40,0.40,0.40}{##1}}}
\expandafter\def\csname PY@tok@mf\endcsname{\def\PY@tc##1{\textcolor[rgb]{0.40,0.40,0.40}{##1}}}
\expandafter\def\csname PY@tok@mh\endcsname{\def\PY@tc##1{\textcolor[rgb]{0.40,0.40,0.40}{##1}}}
\expandafter\def\csname PY@tok@mi\endcsname{\def\PY@tc##1{\textcolor[rgb]{0.40,0.40,0.40}{##1}}}
\expandafter\def\csname PY@tok@il\endcsname{\def\PY@tc##1{\textcolor[rgb]{0.40,0.40,0.40}{##1}}}
\expandafter\def\csname PY@tok@mo\endcsname{\def\PY@tc##1{\textcolor[rgb]{0.40,0.40,0.40}{##1}}}
\expandafter\def\csname PY@tok@ch\endcsname{\let\PY@it=\textit\def\PY@tc##1{\textcolor[rgb]{0.25,0.50,0.50}{##1}}}
\expandafter\def\csname PY@tok@cm\endcsname{\let\PY@it=\textit\def\PY@tc##1{\textcolor[rgb]{0.25,0.50,0.50}{##1}}}
\expandafter\def\csname PY@tok@cpf\endcsname{\let\PY@it=\textit\def\PY@tc##1{\textcolor[rgb]{0.25,0.50,0.50}{##1}}}
\expandafter\def\csname PY@tok@c1\endcsname{\let\PY@it=\textit\def\PY@tc##1{\textcolor[rgb]{0.25,0.50,0.50}{##1}}}
\expandafter\def\csname PY@tok@cs\endcsname{\let\PY@it=\textit\def\PY@tc##1{\textcolor[rgb]{0.25,0.50,0.50}{##1}}}

\def\PYZbs{\char`\\}
\def\PYZus{\char`\_}
\def\PYZob{\char`\{}
\def\PYZcb{\char`\}}
\def\PYZca{\char`\^}
\def\PYZam{\char`\&}
\def\PYZlt{\char`\<}
\def\PYZgt{\char`\>}
\def\PYZsh{\char`\#}
\def\PYZpc{\char`\%}
\def\PYZdl{\char`\$}
\def\PYZhy{\char`\-}
\def\PYZsq{\char`\'}
\def\PYZdq{\char`\"}
\def\PYZti{\char`\~}
% for compatibility with earlier versions
\def\PYZat{@}
\def\PYZlb{[}
\def\PYZrb{]}
\makeatother


    % Exact colors from NB
    \definecolor{incolor}{rgb}{0.0, 0.0, 0.5}
    \definecolor{outcolor}{rgb}{0.545, 0.0, 0.0}



    
    % Prevent overflowing lines due to hard-to-break entities
    \sloppy 
    % Setup hyperref package
    \hypersetup{
      breaklinks=true,  % so long urls are correctly broken across lines
      colorlinks=true,
      urlcolor=urlcolor,
      linkcolor=linkcolor,
      citecolor=citecolor,
      }
    % Slightly bigger margins than the latex defaults
    
    \geometry{verbose,tmargin=1in,bmargin=1in,lmargin=1in,rmargin=1in}
    
    

    \begin{document}
    
    
    \maketitle
    
    

    
    \hypertarget{grab-the-latest-exoplanet-database-from-caltech.}{%
\section{Grab the latest exoplanet database from
CalTech.}\label{grab-the-latest-exoplanet-database-from-caltech.}}

    \begin{Verbatim}[commandchars=\\\{\}]
{\color{incolor}In [{\color{incolor}1}]:} \PY{k+kn}{import} \PY{n+nn}{sys}\PY{o}{,} \PY{n+nn}{os}
        \PY{k+kn}{import} \PY{n+nn}{numpy} \PY{k}{as} \PY{n+nn}{np}
        \PY{k+kn}{import} \PY{n+nn}{urllib} \PY{k}{as} \PY{n+nn}{ul}
        \PY{k+kn}{import} \PY{n+nn}{gwTools} \PY{k}{as} \PY{n+nn}{gwt}
        \PY{k+kn}{import} \PY{n+nn}{pandas} \PY{k}{as} \PY{n+nn}{pd}
\end{Verbatim}


    \hypertarget{save-and-filter-down-to-the-needed-parameters-for-gw-calculations-with-binary-masses.}{%
\subsection{Save and filter down to the needed parameters for GW
calculations with binary
masses.}\label{save-and-filter-down-to-the-needed-parameters-for-gw-calculations-with-binary-masses.}}

    \begin{Verbatim}[commandchars=\\\{\}]
{\color{incolor}In [{\color{incolor}2}]:} \PY{c+c1}{\PYZsh{} The Directory to save the csv file.}
        \PY{n}{thisDir} \PY{o}{=} \PY{n}{os}\PY{o}{.}\PY{n}{getcwd}\PY{p}{(}\PY{p}{)}
        \PY{n}{csvDir} \PY{o}{=} \PY{n}{thisDir} \PY{o}{+} \PY{l+s+s1}{\PYZsq{}}\PY{l+s+s1}{/../dbases/}\PY{l+s+s1}{\PYZsq{}}
\end{Verbatim}


    \begin{Verbatim}[commandchars=\\\{\}]
{\color{incolor}In [{\color{incolor}3}]:} \PY{c+c1}{\PYZsh{} The search URL and search string/request. }
        \PY{n}{exopURL} \PY{o}{=} \PYZbs{}
          \PY{l+s+s2}{\PYZdq{}}\PY{l+s+s2}{https://exoplanetarchive.ipac.caltech.edu/cgi\PYZhy{}bin/nstedAPI/nph\PYZhy{}}\PY{l+s+se}{\PYZbs{}}
        \PY{l+s+s2}{nstedAPI?}\PY{l+s+s2}{\PYZdq{}} \PY{p}{;}
        
        \PY{c+c1}{\PYZsh{}searchString = \PYZbs{}}
        \PY{c+c1}{\PYZsh{}\PYZdq{}table=exoplanets\PYZam{}select=pl\PYZus{}hostname,ra,dec\PYZam{}order=dec\PYZam{}format=CSV\PYZdq{};*)}
        
        \PY{c+c1}{\PYZsh{} The Below does NOT have right ascension and declination.  Will likely want them for further work.}
        \PY{c+c1}{\PYZsh{} Can add later in its own Panda dataframe and/or merge into the main one in GWStrainPlotsSNR.}
        \PY{c+c1}{\PYZsh{} variables come from NASA Exoplanet Archive, the keywords are defined here: }
        \PY{c+c1}{\PYZsh{}https://exoplanetarchive.ipac.caltech.edu/docs/API\PYZus{}exoplanet\PYZus{}columns.html }
        \PY{n}{searchString} \PY{o}{=} \PYZbs{}
          \PY{l+s+s2}{\PYZdq{}}\PY{l+s+s2}{table=exoplanets\PYZam{}select=pl\PYZus{}hostname,pl\PYZus{}letter,pl\PYZus{}discmethod,pl\PYZus{}}\PY{l+s+se}{\PYZbs{}}
        \PY{l+s+s2}{orbper,pl\PYZus{}orbsmax,pl\PYZus{}orbeccen,pl\PYZus{}bmassj,st\PYZus{}dist,st\PYZus{}mass,rowupdate,st\PYZus{}}\PY{l+s+se}{\PYZbs{}}
        \PY{l+s+s2}{plx\PYZam{}order=dec\PYZam{}format=CSV}\PY{l+s+s2}{\PYZdq{}}\PY{p}{;}
\end{Verbatim}


    \begin{Verbatim}[commandchars=\\\{\}]
{\color{incolor}In [{\color{incolor}4}]:} \PY{c+c1}{\PYZsh{} Set to True to re\PYZhy{}read the EXop Dbase from Caltech.  False to use csvFname below.}
        \PY{n}{newImport} \PY{o}{=} \PY{k+kc}{True}\PY{p}{;}
        \PY{c+c1}{\PYZsh{}newImport = False;}
\end{Verbatim}


    \begin{Verbatim}[commandchars=\\\{\}]
{\color{incolor}In [{\color{incolor}5}]:} \PY{c+c1}{\PYZsh{} csv file below was downloaded earlier with code below.  newImport = False to use it.  Or new csv will be}
        \PY{c+c1}{\PYZsh{} created.  This takes a few seconds.}
        \PY{n}{csvFname} \PY{o}{=} \PY{n}{csvDir} \PY{o}{+} \PY{l+s+s1}{\PYZsq{}}\PY{l+s+s1}{exopP\PYZus{}20180408\PYZus{}141319.csv}\PY{l+s+s1}{\PYZsq{}}  
        \PY{k}{if} \PY{n}{newImport}\PY{p}{:}
            \PY{n}{myDateTimeStamp} \PY{o}{=} \PY{n}{gwt}\PY{o}{.}\PY{n}{dateTimeStamp}\PY{p}{(}\PY{p}{)}  \PY{c+c1}{\PYZsh{} See the gwtools.py file with this and other functions in it.}
            \PY{n}{csvFname} \PY{o}{=} \PY{n}{csvDir} \PY{o}{+} \PY{l+s+s1}{\PYZsq{}}\PY{l+s+s1}{exopP\PYZus{}}\PY{l+s+s1}{\PYZsq{}} \PY{o}{+} \PY{n}{myDateTimeStamp} \PY{o}{+} \PY{l+s+s1}{\PYZsq{}}\PY{l+s+s1}{.csv}\PY{l+s+s1}{\PYZsq{}}
            \PY{n}{ofile} \PY{o}{=} \PY{n+nb}{open}\PY{p}{(}\PY{n}{csvFname}\PY{p}{,} \PY{l+s+s1}{\PYZsq{}}\PY{l+s+s1}{w}\PY{l+s+s1}{\PYZsq{}}\PY{p}{)}
            \PY{k}{with} \PY{n}{ul}\PY{o}{.}\PY{n}{request}\PY{o}{.}\PY{n}{urlopen}\PY{p}{(}\PY{n}{exopURL} \PY{o}{+} \PY{n}{searchString}\PY{p}{)} \PY{k}{as} \PY{n}{response}\PY{p}{:}
                \PY{k}{for} \PY{n}{aline} \PY{o+ow}{in} \PY{n}{response}\PY{p}{:}
                    \PY{n}{ofile}\PY{o}{.}\PY{n}{write}\PY{p}{(} \PY{n}{aline}\PY{o}{.}\PY{n}{decode}\PY{p}{(}\PY{l+s+s1}{\PYZsq{}}\PY{l+s+s1}{utf\PYZhy{}8}\PY{l+s+s1}{\PYZsq{}}\PY{p}{)} \PY{p}{)}  \PY{c+c1}{\PYZsh{} byte\PYZhy{}string needs to be decoded. utf\PYZhy{}8 is common encoding}
                    \PY{c+c1}{\PYZsh{}print( aline )}
            \PY{n}{ofile}\PY{o}{.}\PY{n}{close}\PY{p}{(}\PY{p}{)}
        
        \PY{n+nb}{print}\PY{p}{(}\PY{l+s+s1}{\PYZsq{}}\PY{l+s+s1}{Saved database file }\PY{l+s+s1}{\PYZsq{}} \PY{o}{+} \PY{n}{csvFname}\PY{p}{)}
\end{Verbatim}


    \begin{Verbatim}[commandchars=\\\{\}]
Saved database file /home/gabella/Documents/astro/exop/exoplanetsMath/python/../dbases/exopP\_20180412\_093559.csv

    \end{Verbatim}

    \hypertarget{check-the-file-contents-and-number-of-exops-with-required-data-filteringdropping.}{%
\subsection{Check the file contents and number of exops with required
data
(filtering/dropping).}\label{check-the-file-contents-and-number-of-exops-with-required-data-filteringdropping.}}

    \begin{Verbatim}[commandchars=\\\{\}]
{\color{incolor}In [{\color{incolor}6}]:} \PY{c+c1}{\PYZsh{} Checked that it worked by printing a couple of lines.}
        \PY{k}{with} \PY{n+nb}{open}\PY{p}{(}\PY{n}{csvFname}\PY{p}{,} \PY{l+s+s1}{\PYZsq{}}\PY{l+s+s1}{r}\PY{l+s+s1}{\PYZsq{}}\PY{p}{)} \PY{k}{as} \PY{n}{ifile}\PY{p}{:}
            \PY{n+nb}{print}\PY{p}{(}\PY{n}{ifile}\PY{o}{.}\PY{n}{readline}\PY{p}{(}\PY{p}{)}\PY{p}{,} \PY{l+s+s1}{\PYZsq{}}\PY{l+s+se}{\PYZbs{}n}\PY{l+s+s1}{\PYZsq{}}\PY{p}{,} \PY{n}{ifile}\PY{o}{.}\PY{n}{readline}\PY{p}{(}\PY{p}{)} \PY{p}{)}  \PY{c+c1}{\PYZsh{}Print a couple of lines }
        
            \PY{n}{ifile}\PY{o}{.}\PY{n}{seek}\PY{p}{(}\PY{l+m+mi}{0}\PY{p}{)}\PY{p}{;}  \PY{c+c1}{\PYZsh{} and reset the pointer.}
        
            \PY{n}{dbData} \PY{o}{=} \PY{n}{pd}\PY{o}{.}\PY{n}{read\PYZus{}csv}\PY{p}{(}\PY{n}{ifile}\PY{p}{)}  \PY{c+c1}{\PYZsh{} Read the file into a Panda dataframe}
            \PY{n}{ifile}\PY{o}{.}\PY{n}{close}\PY{p}{(}\PY{p}{)}
\end{Verbatim}


    \begin{Verbatim}[commandchars=\\\{\}]
pl\_hostname,pl\_letter,pl\_discmethod,pl\_orbper,pl\_orbsmax,pl\_orbeccen,pl\_bmassj,st\_dist,st\_mass,rowupdate,st\_plx
 
 HD 142022 A,b,Radial Velocity,1928.00000000,3.030000,0.530000,5.10000,35.87,0.99,2014-05-14,27.88


    \end{Verbatim}

    \begin{Verbatim}[commandchars=\\\{\}]
{\color{incolor}In [{\color{incolor}7}]:} \PY{n}{dbData}\PY{o}{.}\PY{n}{head}\PY{p}{(}\PY{l+m+mi}{10}\PY{p}{)}  \PY{c+c1}{\PYZsh{} NaN\PYZsq{}s show up in the field with no data!  Want to drop these from the dataframe.}
\end{Verbatim}


\begin{Verbatim}[commandchars=\\\{\}]
{\color{outcolor}Out[{\color{outcolor}7}]:}    pl\_hostname pl\_letter    pl\_discmethod    pl\_orbper  pl\_orbsmax  \textbackslash{}
        0  HD 142022 A         b  Radial Velocity  1928.000000      3.0300   
        1     HD 39091         b  Radial Velocity  2151.000000      3.3800   
        2  HD 137388 A         b  Radial Velocity   330.000000      0.8900   
        3      GJ 3021         b  Radial Velocity   133.710000      0.4900   
        4     HD 63454         b  Radial Velocity     2.818049      0.0368   
        5    HD 212301         b  Radial Velocity     2.245715      0.0360   
        6      CHXR 73         b          Imaging          NaN    210.0000   
        7       CT Cha         b          Imaging          NaN    440.0000   
        8    HD 196067         b  Radial Velocity  3638.000000      5.0200   
        9     HD 68402         b  Radial Velocity  1103.000000      2.1800   
        
           pl\_orbeccen  pl\_bmassj  st\_dist  st\_mass   rowupdate  st\_plx  
        0       0.5300      5.100    35.87     0.99  2014-05-14   27.88  
        1       0.6405     10.270    18.21     1.10  2014-07-23   54.92  
        2       0.3600      0.223    38.45     0.86  2014-05-14   26.01  
        3       0.5110      3.370    17.62     0.90  2014-05-14   56.76  
        4       0.0000      0.398    35.80     0.84  2015-03-26   27.93  
        5       0.0000      0.450    52.72     1.27  2014-05-14   18.97  
        6          NaN     12.569      NaN     0.35  2014-05-14     NaN  
        7          NaN     17.000   165.00      NaN  2014-05-14     NaN  
        8       0.6600      6.900    43.57     1.29  2014-05-14   22.95  
        9       0.0300      3.070    78.00     1.12  2016-11-10   12.82  
\end{Verbatim}
            
    \begin{Verbatim}[commandchars=\\\{\}]
{\color{incolor}In [{\color{incolor}8}]:} \PY{c+c1}{\PYZsh{} Graphic from the Mathematica file for filtering, data from 20180201\PYZus{}092937.csv.}
        \PY{n+nb}{print}\PY{p}{(}\PY{l+s+s1}{\PYZsq{}}\PY{l+s+s1}{Screenshot from the Mathematica file for filtering, data from 20180201\PYZus{}092937.csv.}\PY{l+s+s1}{\PYZsq{}}\PY{p}{)}
        \PY{n+nb}{print}\PY{p}{(}\PY{l+s+s1}{\PYZsq{}}\PY{l+s+s1}{For comparison with below .dropna() Pandas version.}\PY{l+s+s1}{\PYZsq{}}\PY{p}{)}
        \PY{k+kn}{from} \PY{n+nn}{IPython}\PY{n+nn}{.}\PY{n+nn}{display} \PY{k}{import} \PY{n}{Image}
        \PY{n}{Image}\PY{p}{(}\PY{n}{filename} \PY{o}{=} \PY{l+s+s1}{\PYZsq{}}\PY{l+s+s1}{filteringMathematica.png}\PY{l+s+s1}{\PYZsq{}}\PY{p}{)}
\end{Verbatim}


    \begin{Verbatim}[commandchars=\\\{\}]
Screenshot from the Mathematica file for filtering, data from 20180201\_092937.csv.
For comparison with below .dropna() Pandas version.

    \end{Verbatim}
\texttt{\color{outcolor}Out[{\color{outcolor}8}]:}
    
    \begin{center}
    \adjustimage{max size={0.9\linewidth}{0.9\paperheight}}{output_10_1.png}
    \end{center}
    { \hspace*{\fill} \\}
    

    \begin{Verbatim}[commandchars=\\\{\}]
{\color{incolor}In [{\color{incolor}9}]:} \PY{n+nb}{print}\PY{p}{(}\PY{l+s+s1}{\PYZsq{}}\PY{l+s+s1}{Original number of exoplanets in csv file }\PY{l+s+s1}{\PYZsq{}}\PY{p}{,} \PY{n+nb}{len}\PY{p}{(}\PY{n}{dbData}\PY{p}{)} \PY{p}{)}
\end{Verbatim}


    \begin{Verbatim}[commandchars=\\\{\}]
Original number of exoplanets in csv file  3711

    \end{Verbatim}

    \hypertarget{use-.dropna-in-pandas-to-remove-all-the-nans-i.e.the-missing-data-rowsexops.}{%
\subsection{Use .dropna() in pandas to remove all the NaN's, i.e.~the
missing data
rows/exops.}\label{use-.dropna-in-pandas-to-remove-all-the-nans-i.e.the-missing-data-rowsexops.}}

    \begin{Verbatim}[commandchars=\\\{\}]
{\color{incolor}In [{\color{incolor}10}]:} \PY{c+c1}{\PYZsh{} \PYZob{}\PYZdq{}pl\PYZus{}hostname\PYZdq{}, \PYZdq{}pl\PYZus{}letter\PYZdq{}, \PYZdq{}pl\PYZus{}discmethod\PYZdq{}, \PYZdq{}pl\PYZus{}orbper\PYZdq{}, \PYZbs{}}
         \PY{c+c1}{\PYZsh{} \PYZdq{}pl\PYZus{}orbsmax\PYZdq{}, \PYZdq{}pl\PYZus{}orbeccen\PYZdq{}, \PYZdq{}pl\PYZus{}bmassj\PYZdq{}, \PYZdq{}st\PYZus{}dist\PYZdq{}, \PYZdq{}st\PYZus{}mass\PYZdq{}, \PYZbs{}}
         \PY{c+c1}{\PYZsh{} \PYZdq{}rowupdate\PYZdq{}, \PYZdq{}st\PYZus{}plx\PYZdq{}\PYZcb{}}
         \PY{n+nb}{print}\PY{p}{(}\PY{l+s+s1}{\PYZsq{}}\PY{l+s+s1}{Rows of all data, dbData }\PY{l+s+s1}{\PYZsq{}}\PY{p}{,} \PY{n+nb}{len}\PY{p}{(}\PY{n}{dbData}\PY{p}{)} \PY{p}{)}
         \PY{n}{aData} \PY{o}{=} \PY{n}{dbData}\PY{o}{.}\PY{n}{dropna}\PY{p}{(}\PY{n}{axis} \PY{o}{=} \PY{l+m+mi}{0}\PY{p}{,} \PY{n}{how} \PY{o}{=} \PY{l+s+s1}{\PYZsq{}}\PY{l+s+s1}{any}\PY{l+s+s1}{\PYZsq{}}\PY{p}{,} \PY{n}{subset} \PY{o}{=} \PY{p}{[}\PY{l+s+s1}{\PYZsq{}}\PY{l+s+s1}{pl\PYZus{}orbeccen}\PY{l+s+s1}{\PYZsq{}}\PY{p}{]}\PY{p}{)}
         \PY{n+nb}{print}\PY{p}{(}\PY{l+s+s1}{\PYZsq{}}\PY{l+s+s1}{Rows with pl\PYZus{}orbeccen}\PY{l+s+se}{\PYZbs{}t}\PY{l+s+s1}{\PYZsq{}}\PY{p}{,} \PY{n+nb}{len}\PY{p}{(}\PY{n}{aData}\PY{p}{)} \PY{p}{)}
         \PY{n}{aData} \PY{o}{=} \PY{n}{aData}\PY{o}{.}\PY{n}{dropna}\PY{p}{(}\PY{n}{axis} \PY{o}{=} \PY{l+m+mi}{0}\PY{p}{,} \PY{n}{how} \PY{o}{=} \PY{l+s+s1}{\PYZsq{}}\PY{l+s+s1}{any}\PY{l+s+s1}{\PYZsq{}}\PY{p}{,} \PY{n}{subset} \PY{o}{=} \PY{p}{[}\PY{l+s+s1}{\PYZsq{}}\PY{l+s+s1}{pl\PYZus{}orbper}\PY{l+s+s1}{\PYZsq{}}\PY{p}{]}\PY{p}{)}
         \PY{n+nb}{print}\PY{p}{(}\PY{l+s+s1}{\PYZsq{}}\PY{l+s+s1}{Rows with pl\PYZus{}orbper}\PY{l+s+se}{\PYZbs{}t}\PY{l+s+s1}{\PYZsq{}}\PY{p}{,} \PY{n+nb}{len}\PY{p}{(}\PY{n}{aData}\PY{p}{)} \PY{p}{)}
         \PY{n}{aData} \PY{o}{=} \PY{n}{aData}\PY{o}{.}\PY{n}{dropna}\PY{p}{(}\PY{n}{axis} \PY{o}{=} \PY{l+m+mi}{0}\PY{p}{,} \PY{n}{how} \PY{o}{=} \PY{l+s+s1}{\PYZsq{}}\PY{l+s+s1}{any}\PY{l+s+s1}{\PYZsq{}}\PY{p}{,} \PY{n}{subset} \PY{o}{=} \PY{p}{[}\PY{l+s+s1}{\PYZsq{}}\PY{l+s+s1}{pl\PYZus{}orbsmax}\PY{l+s+s1}{\PYZsq{}}\PY{p}{]}\PY{p}{)}
         \PY{n+nb}{print}\PY{p}{(}\PY{l+s+s1}{\PYZsq{}}\PY{l+s+s1}{Rows with pl\PYZus{}orbsmax}\PY{l+s+se}{\PYZbs{}t}\PY{l+s+s1}{\PYZsq{}}\PY{p}{,} \PY{n+nb}{len}\PY{p}{(}\PY{n}{aData}\PY{p}{)} \PY{p}{)}
         \PY{n}{aData} \PY{o}{=} \PY{n}{aData}\PY{o}{.}\PY{n}{dropna}\PY{p}{(}\PY{n}{axis} \PY{o}{=} \PY{l+m+mi}{0}\PY{p}{,} \PY{n}{how} \PY{o}{=} \PY{l+s+s1}{\PYZsq{}}\PY{l+s+s1}{any}\PY{l+s+s1}{\PYZsq{}}\PY{p}{,} \PY{n}{subset} \PY{o}{=} \PY{p}{[}\PY{l+s+s1}{\PYZsq{}}\PY{l+s+s1}{pl\PYZus{}bmassj}\PY{l+s+s1}{\PYZsq{}}\PY{p}{]}\PY{p}{)}
         \PY{n+nb}{print}\PY{p}{(}\PY{l+s+s1}{\PYZsq{}}\PY{l+s+s1}{Rows with pl\PYZus{}bmassj}\PY{l+s+se}{\PYZbs{}t}\PY{l+s+s1}{\PYZsq{}}\PY{p}{,} \PY{n+nb}{len}\PY{p}{(}\PY{n}{aData}\PY{p}{)} \PY{p}{)}
         \PY{n}{aData} \PY{o}{=} \PY{n}{aData}\PY{o}{.}\PY{n}{dropna}\PY{p}{(}\PY{n}{axis} \PY{o}{=} \PY{l+m+mi}{0}\PY{p}{,} \PY{n}{how} \PY{o}{=} \PY{l+s+s1}{\PYZsq{}}\PY{l+s+s1}{any}\PY{l+s+s1}{\PYZsq{}}\PY{p}{,} \PY{n}{subset} \PY{o}{=} \PY{p}{[}\PY{l+s+s1}{\PYZsq{}}\PY{l+s+s1}{st\PYZus{}dist}\PY{l+s+s1}{\PYZsq{}}\PY{p}{]}\PY{p}{)}
         \PY{n+nb}{print}\PY{p}{(}\PY{l+s+s1}{\PYZsq{}}\PY{l+s+s1}{Rows with st\PYZus{}dist}\PY{l+s+se}{\PYZbs{}t}\PY{l+s+s1}{\PYZsq{}}\PY{p}{,} \PY{n+nb}{len}\PY{p}{(}\PY{n}{aData}\PY{p}{)} \PY{p}{)}
         \PY{n}{aData} \PY{o}{=} \PY{n}{aData}\PY{o}{.}\PY{n}{dropna}\PY{p}{(}\PY{n}{axis} \PY{o}{=} \PY{l+m+mi}{0}\PY{p}{,} \PY{n}{how} \PY{o}{=} \PY{l+s+s1}{\PYZsq{}}\PY{l+s+s1}{any}\PY{l+s+s1}{\PYZsq{}}\PY{p}{,} \PY{n}{subset} \PY{o}{=} \PY{p}{[}\PY{l+s+s1}{\PYZsq{}}\PY{l+s+s1}{st\PYZus{}mass}\PY{l+s+s1}{\PYZsq{}}\PY{p}{]}\PY{p}{)}
         \PY{n+nb}{print}\PY{p}{(}\PY{l+s+s1}{\PYZsq{}}\PY{l+s+s1}{Rows with st\PYZus{}mass}\PY{l+s+se}{\PYZbs{}t}\PY{l+s+s1}{\PYZsq{}}\PY{p}{,} \PY{n+nb}{len}\PY{p}{(}\PY{n}{aData}\PY{p}{)} \PY{p}{)}
\end{Verbatim}


    \begin{Verbatim}[commandchars=\\\{\}]
Rows of all data, dbData  3711
Rows with pl\_orbeccen	 1172
Rows with pl\_orbper	 1172
Rows with pl\_orbsmax	 1107
Rows with pl\_bmassj	 1027
Rows with st\_dist	 920
Rows with st\_mass	 910

    \end{Verbatim}


    % Add a bibliography block to the postdoc
    
    
    
    \end{document}
